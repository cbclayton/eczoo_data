% Here is an example of a table that you can export
% as vector graphics.
%
% COMPILATION INSTRUCTIONS:
%
% >   lualatex table_template.tex
% >   dvisvgm --pdf table_template.pdf
%
% (If you're on Mac OS, you might have to do "brew install ghostscript" and add
% the option "--libgs=/usr/local/opt/ghostscript/lib/libgs.dylib" to the dvisvgm
% call)
%
%
% SCROLL DOWN TO EDIT THE TABLE (search for "TABLE STARTS HERE")
%
\documentclass[10pt]{article}

\usepackage[table]{xcolor}

% please compile with lualatex
\usepackage{fontspec}

\setsansfont{Source Sans Pro}
\renewcommand{\familydefault}{\sfdefault}

%
% --- begin custom preamble ---
%
\usepackage{tabularx}
\usepackage{booktabs}

\usepackage{amsmath}
\usepackage{amsthm}

\usepackage{phfparen}
\usepackage{phfqit}

%
% --- end custom preamble ---
%
%
% --- begin set up {preview} ---
%
\usepackage[active,delayed,tightpage]{preview}

\pagestyle{empty}
%
% --- end set up {preview} ---
%

\begin{document}
%
\textwidth=500pt\relax
\hsize=\textwidth\relax
\parindent=0pt\relax
\renewcommand{\arraystretch}{1.2}
\newcolumntype{C}{>{\centering\arraybackslash}X}
%
\begin{preview}%
%
%
% TABLE STARTS HERE --->
%
%
\begin{tabular}{c|ccccccc}
% If you use tabularx, try to use \textwidth if possible:
%\begin{tabularx}{\textwidth}{XCCC}
\toprule
% Header
 $S_q$ &
$\infty$ & $0$ & $1$ & $\dots$ & $j$ & $\dots$ & $p-1$
\\
% Body
\midrule
% -----
$\infty$ &
$0$  & $1$  & $1$ & $\dots$ & $1$ & $\dots$ & $1$
\\ 
$0$ &
$\chi(-1)$ & $\chi(0)$ & $\chi(1)$ & $\dots$ & $\chi(j)$ & $\dots$ & $\chi(p-1)$
\\
$1$
& $\chi(-1)$
\\
$\vdots$ & $\vdots$ &&& $\ddots$ && $\ddots$
\\
$i$ & $\chi(-1)$ &&&& $\chi(j-i)$ \\
$\vdots$ & $\vdots$ &&& $\ddots$ && $\ddots$ \\
$p-1$ & $\chi(-1)$ \\
% -----
\bottomrule
\end{tabular}%
%
% <--- TABLE ENDED HERE
%
%
%
\end{preview}%
\end{document}

%%% Local Variables:
%%% mode: lualatex
%%% TeX-master: t
%%% End: